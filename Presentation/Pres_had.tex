\documentclass{beamer}

%Rajouter les slides de photoélasticité à fin

\input{preamble_tikz.tex}

\usepackage[utf8]{inputenc}
\usepackage[english]{babel}
\usepackage{animate}
\usepackage[absolute,overlay]{textpos}

\usepackage{ulem}

\usepackage{amssymb}
\usepackage{psfrag}
\usepackage[utf8]{inputenc}
\usepackage{amsmath}
\usepackage{amsfonts}
\usepackage{amssymb}
\usepackage{graphicx}
\usepackage{subcaption}
\usepackage{fancyhdr}
\usepackage{multicol}
\usepackage{eurosym} % symbole €
\usepackage{siunitx}
\usepackage{stmaryrd}
\usepackage{bm}

\everymath{\displaystyle}

\usepackage{bm}
\setbeamertemplate{navigation symbols}{}
\setbeamercolor{structure}{fg=pink!250!white}

\usetheme{Warsaw}
\setbeamertemplate{headline}{}
\addtobeamertemplate{footline}{\ \ \ \ \ \insertframenumber/\inserttotalframenumber\hspace{2em}\null}

\title{Yolo}
\author{Félix Bunel et Hadrien Vergnet}
\titlegraphic{\includegraphics[height=1.7cm]{figures/sdm.png}} 
\date{16/01/2017}

\usefonttheme[onlymath]{serif}

%%%%%%%%%%%%%%%%%%%%%%%%%%%%%%%%%%%%%%%%%%%%%%%%%%%%%%%%%%%%%%%%%%%%%%%%%%%%%%%%%%%%%
\begin{document}
%%%%%%%%%%%%%%%%%%%%%%%%%%%%%%%%%%%%%%%%%%%%%%%%%%%%%%%%%%%%%%%%%%%%%%%%%%%%%%%%%%%%%

%%%%%%%%%%%%% Slide de garde
\begin{frame}[plain]

\begin{columns}
\begin{column}{3.3cm}
\center
   \includegraphics[height=1.3cm]{figures/logo_lyon1.jpg}
\end{column}
\begin{column}{3.3cm}
\center
\includegraphics[height=1.3cm]{figures/logo_ens.jpg}
\end{column}
\begin{column}{3.3cm}
\center
\includegraphics[height=1.3cm]{figures/logo_univ_lyon.jpg}
\end{column}
\end{columns}

\titlepage
\end{frame}
%%%%%%%%%%%%%

\title{yolo}
%%%%%%%%%%%%%
\begin{frame}
\frametitle{Introduction}
\framesubtitle{\ }

\end{frame}
%%%%%%%%%%%%%


%%%%%%%%%%%%% Slide de sommaire
\begin{frame}
	\frametitle{Sommaire}
	\framesubtitle{\ }
	\tableofcontents
\end{frame}
%%%%%%%%%%%%%

\section{section}
\subsection{soussection}

%%%%%%%%%%%%% 
\begin{frame}{Liquid crystal display and Fréedericksz transition}
    \begin{figure}[h!]
    \center
        \begin{subfigure}[b]{0.49\textwidth}
        	\center
            \begin{tikzpicture}[scale=0.8]
                \node at (0,0) {\includegraphics[scale=1.2]{figures/freedericksz_a.pdf}};
                \node at (-2.35,1.25) {$x$}; 
                \node at(-1.2,1.5) {$y$};
                \node at (-2.1,2.4) {$z$};
                \node at (-1.55,0.6) {$\vec{U}^\star$}; 
            \end{tikzpicture} 
        	\caption{Pixel "off" state}
        	\label{allume}
        \end{subfigure}	
        \begin{subfigure}[b]{0.49\textwidth}
        	\center
            \begin{tikzpicture}[scale=0.8]
                \node at (0,0) {\includegraphics[scale=1.2]{figures/freedericksz_b.pdf}};
                \node at (-1.6,2.2) {$\vec{U}^\star$}; 
            \end{tikzpicture} 
        	\caption{Pixel "on" state}
        	\label{etteint}
        \end{subfigure}	
        
        \label{freedericksz} 
    \end{figure}

\center
Diagram of a LCD pixel using \textit{twisted nematic} technology.
\end{frame}
%%%%%%%%%%%%%

%%%%%%%%%%%%% 
\begin{frame}{Molecules orientation}
\framesubtitle{\ }
    \begin{figure}[h!]
    \includegraphics[scale=0.45]{figures/lcd_allume_T01.pdf}
    \end{figure}

\center
Molecule orientation \textbf{without} electric field
\end{frame}
%%%%%%%%%%%%%

%%%%%%%%%%%%% 
\begin{frame}{Molecules orientation}
\framesubtitle{\ }
    \begin{figure}[h!]
    \includegraphics[scale=0.45]{figures/lcd_etteind.pdf}
    \end{figure}
\center
Molecule orientation \textbf{with} electric field
\end{frame}
%%%%%%%%%%%%%


%%%%%%%%%%%%% 
\begin{frame}
	\frametitle{Conclusion}
	\framesubtitle{\ }
    \only<2->{Detailed study of nematic-isotropic transition with Lebwohl-Laser model :
    \begin{itemize}
    \item first order transition at  $T^\star = 1.1232 \pm 0.0005$
    \end{itemize}}
    \only<2>{\center \includegraphics[scale=0.4]{figures/local_energie.pdf}}
    \only<3->{
    Electric field influence :
    \begin{itemize}
        \item shifts transition temperature and critical point for strong fields
    \end{itemize}
    }
    \only<3>{\center \includegraphics[scale=0.33]{figures/electricField_calo.pdf}}
    \only<4->{
    LCD and Fréedericksz :
    \begin{itemize}
        \item Lebwohl-Laser model can be used to model a LCD pixel
    \end{itemize}
    }
    \only<4>{
    \begin{figure}[h!]
    \center
        \begin{subfigure}[b]{0.49\textwidth}
        	\center
            \begin{tikzpicture}[scale=0.45]
                \node at (0,0) {\includegraphics[scale=0.65]{figures/freedericksz_a.pdf}};
                \node at (-2.35,1.25) {$x$}; 
                \node at(-1.2,1.5) {$y$};
                \node at (-2.1,2.4) {$z$};
                \node at (-1.55,0.6) {$\vec{U}^\star$}; 
            \end{tikzpicture} 
        	\label{allume}
        \end{subfigure}	
        \begin{subfigure}[b]{0.49\textwidth}
        	\center
            \begin{tikzpicture}[scale=0.45]
                \node at (0,0) {\includegraphics[scale=0.65]{figures/freedericksz_b.pdf}};
                \node at (-1.6,2.2) {$\vec{U}^\star$}; 
            \end{tikzpicture} 
        	\label{etteint}
        \end{subfigure}	
        
        \label{freedericksz} 
    \end{figure}
    }


\end{frame}
%%%%%%%%%%%%%

%%%%%%%%%%%%% 
\begin{frame}
	\frametitle{Perspectives}
	\framesubtitle{\ }
    Perspectives :
    \begin{itemize}
        \item Find the value of the critical field in that Fréedericksz transition 
        \item Study the temperature dependence of that transition
    \end{itemize}

\end{frame}
%%%%%%%%%%%%%

%%%%%%%%%%%%% 
\begin{frame}
	\frametitle{\ }
	\framesubtitle{\ }

{\center $\qquad$ \Huge Merci pour votre attention}

\end{frame}
%%%%%%%%%%%%%

%%%%%%%%%%%%% 
\begin{frame}
	\frametitle{yolo}
	\framesubtitle{\ }


\end{frame}
%%%%%%%%%%%%%


%%%%%%%%%%%%%
\end{document}